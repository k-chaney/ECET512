\documentclass{article}

\usepackage{fancyhdr}
\usepackage{extramarks}
\usepackage{amsmath}
\usepackage{amsthm}
\usepackage{amsfonts}
\usepackage{tikz}
\usepackage[plain]{algorithm}
\usepackage{algpseudocode}

\usetikzlibrary{automata,positioning}

%
% Basic Document Settings
%

\topmargin=-0.45in
\evensidemargin=0in
\oddsidemargin=0in
\textwidth=6.5in
\textheight=9.0in
\headsep=0.25in

\linespread{1.1}

\pagestyle{fancy}
\lhead{\hmwkAuthorInitials}
\chead{\hmwkClass\ (\hmwkClassInstructor\ \hmwkClassTime): \hmwkTitle}
\rhead{\firstxmark}
\lfoot{\lastxmark}
\cfoot{\thepage}

\renewcommand\headrulewidth{0.4pt}
\renewcommand\footrulewidth{0.4pt}

\setlength\parindent{0pt}

%
% Create Problem Sections
%

\newcommand{\enterProblemHeader}[1]{
    \nobreak\extramarks{}{Problem \arabic{#1} continued on next page\ldots}\nobreak{}
    \nobreak\extramarks{Problem \arabic{#1} (continued)}{Problem \arabic{#1} continued on next page\ldots}\nobreak{}
}

\newcommand{\exitProblemHeader}[1]{
    \nobreak\extramarks{Problem \arabic{#1} (continued)}{Problem \arabic{#1} continued on next page\ldots}\nobreak{}
    \stepcounter{#1}
    \nobreak\extramarks{Problem \arabic{#1}}{}\nobreak{}
}

\setcounter{secnumdepth}{0}
\newcounter{partCounter}
\newcounter{homeworkProblemCounter}
\setcounter{homeworkProblemCounter}{1}
\nobreak\extramarks{Problem \arabic{homeworkProblemCounter}}{}\nobreak{}

%
% Homework Problem Environment
%
% This environment takes an optional argument. When given, it will adjust the
% problem counter. This is useful for when the problems given for your
% assignment aren't sequential. See the last 3 problems of this template for an
% example.
%
\newenvironment{homeworkProblem}[1][-1]{
    \ifnum#1>0
        \setcounter{homeworkProblemCounter}{#1}
    \fi
    \section{Problem \arabic{homeworkProblemCounter}}
    \setcounter{partCounter}{1}
    \enterProblemHeader{homeworkProblemCounter}
}{
    \exitProblemHeader{homeworkProblemCounter}
}

%
% Homework Details
%   - Title
%   - Due date
%   - Class
%   - Section/Time
%   - Instructor
%   - Author
%

\newcommand{\hmwkTitle}{Homework 1}
\newcommand{\hmwkDueDate}{October 2^{nd}, 2014}
\newcommand{\hmwkClass}{ECES 511}
\newcommand{\hmwkClassTime}{Section 001}
\newcommand{\hmwkClassInstructor}{Professor Yon Visell}
\newcommand{\hmwkAuthorName}{Kenneth Chaney and Olaniyi Jinadu}
\newcommand{\hmwkAuthorInitials}{KC and OJ}
%
% Title Page
%

\title{
    \vspace{2in}
    \textmd{\textbf{\hmwkClass:\ \hmwkTitle}}\\
    \normalsize\vspace{0.1in}\small{Due\ on\ \hmwkDueDate\ at\ 1830}\\
    \vspace{0.1in}\large{\textit{\hmwkClassInstructor\ \hmwkClassTime}}
    \vspace{3in}
}

\author{\textbf{\hmwkAuthorName}}
\date{}

\renewcommand{\part}[1]{\textbf{\large Part \Alph{partCounter}}\stepcounter{partCounter}\\}

%
% Various Helper Commands
%

% Useful for algorithms
\newcommand{\alg}[1]{\textsc{\bfseries \footnotesize #1}}

% For derivatives
\newcommand{\deriv}[1]{\frac{\mathrm{d}}{\mathrm{d}x} (#1)}

% For partial derivatives
\newcommand{\pderiv}[2]{\frac{\partial}{\partial #1} (#2)}

% Integral dx
\newcommand{\dx}{\mathrm{d}x}

% Alias for the Solution section header
\newcommand{\solution}{\textbf{\large Solution}}

% Probability commands: Expectation, Variance, Covariance, Bias
\newcommand{\E}{\mathrm{E}}
\newcommand{\Var}{\mathrm{Var}}
\newcommand{\Cov}{\mathrm{Cov}}
\newcommand{\Bias}{\mathrm{Bias}}

\begin{document}
% Put make title back in when you can
%\maketitle
\pagebreak
\begin{homeworkProblem}
\begin{equation}\label{eq:block1}
	\begin{bmatrix} \dot x_1 \\ \dot x_2 \end{bmatrix} = \begin{bmatrix} 1 & 0 \\ -1 & 2 \end{bmatrix} \begin{bmatrix} x_1 \\ x_2 \end{bmatrix}+\begin{bmatrix} 4 \\ 1 \end{bmatrix} u
\end{equation}
\begin{equation}\label{eq:block2}
	y = \begin{bmatrix}  1 & 3 \end{bmatrix} \begin{bmatrix} x_1 \\ x_2 \end{bmatrix}
\end{equation}

Equations \ref{eq:block1} and \ref{eq:block2} can be represented as the following block diagram.\\


\end{homeworkProblem}
\pagebreak
\begin{homeworkProblem}
Equations initially given with the problem (manipulated for use later):

\begin{equation}\label{eq:newton}
\begin{aligned}
ml^2\ddot\theta=mgl\sin\theta-b\dot\theta+T \\ 
\ddot\theta=\dfrac{g\sin\theta}{l}-\dfrac{b\dot\theta}{ml^2}+\dfrac{T}{ml^2}
\end{aligned}
\end{equation}\\
\begin{equation}\label{eq:tsat}
T = sat(u)
\end{equation}
\begin{equation}\label{eq:yt}
y = \theta 
\end{equation}

\textbf{Part A}--when \( \theta = 0 \)\\

\begin{equation}
y = \delta \theta_1
\end{equation}

\begin{equation}
\delta \dot \theta = \dfrac{g \cos \theta}{l}\Big|_{\theta=0} * \delta \theta_1 - \dfrac{b \delta \theta_2}{ml^2}+\dfrac{T}{ml^2}
\end{equation}

\begin{equation}\label{eq:linearNM}
\delta \dot \theta = \dfrac{g \delta\theta_1}{l}-\dfrac{b\delta\theta_2}{ml^2}+\dfrac{T}{ml^2}
\end{equation}
Converting equation \ref{eq:linearNM} to linear state space form allows us to observe the system as a whole easily.
\begin{equation}\label{eq:linearM}
\begin{aligned}
\delta \dot \theta = \begin{bmatrix}0&1\\\dfrac{g}{l} & -\dfrac{b}{ml^2}\end{bmatrix} \delta\theta + \begin{bmatrix} 0 \\ \dfrac{1}{ml^2}\end{bmatrix} T \\
y = \begin{bmatrix}1 & 0 \end{bmatrix} \delta \theta
\end{aligned}
\end{equation}

From equation \ref{eq:linearM} you can define the matrices \(A,B,C\) as:

\[
    A = \begin{bmatrix} 0 & 1 \\ \dfrac{g}{l} & -\dfrac{b}{ml^2} \end{bmatrix} , \ 
    B = \begin{bmatrix} 0 \\ \dfrac{1}{ml^2} \end{bmatrix} , \
    C = \begin{bmatrix} 1 & 0 \end{bmatrix} \
\]

\textbf{Part B}--when \( \theta = \pi \)\\
\begin{equation}
\dfrac{g}{l} \cos \theta \Big|_{\theta=\pi} = -\dfrac{g}{l}
\end{equation}

\begin{equation}
\delta\dot\theta = -\dfrac{g}{l}\delta\theta_1-\dfrac{b\delta\theta_2}{ml^2}+\dfrac{T}{ml^2}
\end{equation}

\begin{equation}\label{eq:thetaPi}
\begin{aligned}
\delta\dot\theta = \begin{bmatrix} 0 & 1 \\ - \dfrac{g}{l} & -\dfrac{b}{ml^2} \end{bmatrix} \delta\theta+ \begin{bmatrix} 0 \\ \dfrac{1}{ml^2} \end{bmatrix} T\\
y = \begin{bmatrix} 1 & 0 \end{bmatrix} \delta\theta
\end{aligned}
\end{equation}

From equation \ref{eq:thetaPi} you can define the matrices \(A,B,C\) as:

\[
    A = \begin{bmatrix} 0 & 1 \\ - \dfrac{g}{l} & -\dfrac{b}{ml^2} \end{bmatrix} , \ 
    B = \begin{bmatrix} 0 \\ \dfrac{1}{ml^2} \end{bmatrix} , \
    C = \begin{bmatrix} 1 & 0 \end{bmatrix} \
\]

\end{homeworkProblem}
\pagebreak
\begin{homeworkProblem}

\[
    A_1 = \begin{bmatrix} 3 & 6 \\ 1 & 4 \end{bmatrix} , \ 
    A_2 = \begin{bmatrix} 0 & 1 & -1 \\ 1 & 0 & -1 \\ 1 & 1 & -2 \end{bmatrix}
\]

The definition of the eigenvalues are:

\begin{equation} \label{eq:eigDef}
\det(A - \lambda I) = 0
\end{equation}

Using equation \ref{eq:eigDef} to solve Matrix \(A_1\) we get:


\begin{equation}\label{eq:eigMat1}
\begin{aligned}
\det \left( \begin{bmatrix} 3 & 6 \\ 1 & 4 \end{bmatrix} - \lambda I \right) = 0 \\
\begin{vmatrix} 3-\lambda  & 6 \\ 1 & 4-\lambda \end{vmatrix} = 0
\end{aligned}
\end{equation}

~\\
The solution to a 2-by-2 determinate is expressed as:

\begin{equation*}
ad-bc
\end{equation*}

A polynomial can be formed from the determinate in equation \ref{eq:eigMat1}. 

\begin{equation} \label{eq:poly1}
(3-\lambda)(4-\lambda)-6=0
\end{equation}

Solving equation \ref{eq:poly1} will give us our eigenvalues \(\lambda_1, \lambda_2\) .
\begin{equation}
\begin{aligned}
\lambda^2-7\lambda+6=0 \\
(\lambda-1)(\lambda-6)=0\\
\lambda_1 = 1, \lambda_2 = 6
\end{aligned}
\end{equation}

The same approach can be used for \(A_2\), since that's the case detail for each step will not be given.

\begin{equation}
\begin{aligned}
\det \left( \begin{bmatrix} 0 & 1 & -1 \\ 1 & 0 & -1 \\ 1 & 1 & -2 \end{bmatrix} - \lambda I \right) = 0 \\
\begin{vmatrix} -\lambda & 1 & -1 \\ 1 & -\lambda & -1 \\ 1 & 1 & -2-\lambda \end{vmatrix}=0\\
-\lambda \begin{vmatrix}  -\lambda & -1 \\ 1 & -2-\lambda \end{vmatrix}   - 1 \begin{vmatrix}  1 &  -1 \\ 1 & -2-\lambda \end{vmatrix}  - 1 \begin{vmatrix}  1 & -\lambda \\ 1 & 1 \end{vmatrix} = 0 \\ 
-\lambda(2\lambda+\lambda^2+1)-1(1+\lambda)-1(-2-\lambda+1)= 0\\
-\lambda^3-2\lambda^2-\lambda=0\\
\lambda(\lambda^2+2\lambda=0\\
\lambda(\lambda+1)^2=0\\
\lambda_1 = -1, \lambda_2 = -1, \lambda_3 = 0
\end{aligned}
\end{equation}\\

The eigenvalues of \(A_1\) are \(\lambda_1 = 1, \lambda_2 = 6\) and the eigenvalues of \(A_2\) are \(\lambda_1 = -1, \lambda_2 = -1, \lambda_3 = 0\)

\end{homeworkProblem}


\pagebreak
\begin{homeworkProblem}
The voltage V can be described as the voltage across each element summed together.
\begin{equation}\label{eq:voltBase}
V=V_R+V_L
\end{equation}

\begin{equation}\label{eq:resBase}
V_R=IR
\end{equation}
\begin{equation}\label{eq:inductBase}
V_L=L \dfrac{dI}{dt}
\end{equation}

Substituting equations \ref{eq:resBase} and \ref{eq:inductBase} into equation \ref{eq:voltBase} results in the differential equation that governs the system.


\begin{equation}\label{eq:govern}
\begin{aligned}
V=IR+L \dfrac{dI}{dt}\\
\dfrac{dI}{dt}=-\dfrac{IR}{L}+\dfrac{V}{L}
\end{aligned}
\end{equation}

Utilizing the relationship between current and charge you can use it to augment equation \ref{eq:govern}; that relationship is:
\begin{equation}\label{eq:chargeToCur}
I = \dfrac{dq}{dt}
\end{equation}

Choosing proper state variables \( z_1 = q, z_2 = \dot q \) yields the following first order differential equations

\begin{equation}
\begin{aligned}
& \dot z_1 = z_2 \\ 
& \dot z_2 = -\dfrac{Rz_2}{L}+\dfrac{V}{L}
\end{aligned}
\end{equation}

This set of state space equations can be expressed in matrix form as:

\begin{equation} \label{eq:matrixForm}
\begin{bmatrix} \dot z_1 \\ \dot z_2  \end{bmatrix} = \begin{bmatrix} 0 & 1 \\ 0 & -\dfrac{R}{L}  \end{bmatrix} \begin{bmatrix} z_1 \\ z_2  \end{bmatrix}+\begin{bmatrix} 0 \\ \dfrac{V}{L} \end{bmatrix}
\end{equation}

Using equation \ref{eq:matrixForm} and \ref{eq:chargeToCur} it is possible to write the resulting system in linear state space form.

\begin{equation} \label{eq:stateSpace}
\begin{aligned}
\dot x(t) = \begin{bmatrix} 0 & 1 \\ 0 & -\dfrac{R}{L}  \end{bmatrix} x(t)+\begin{bmatrix} 0 \\ \dfrac{1}{L} \end{bmatrix} u(t) \\ 
y(t) = \begin{bmatrix} 0 & 0 \\ 0 & 1 \end{bmatrix} x(t) + \begin{bmatrix} 0 \\ 0 \end{bmatrix} u(t)
\end{aligned}
\end{equation}

From equation \ref{eq:stateSpace} you can define the matrices \(A,B,C,D\) as:

\[
    A = \begin{bmatrix} 0 & 1 \\ 0 & -\dfrac{R}{L} \end{bmatrix} , \ 
    B = \begin{bmatrix} 0 \\ \dfrac{1}{L} \end{bmatrix} , \
    C = \begin{bmatrix} 0 & 0 \\ 0 & 1 \end{bmatrix} , \
    D = \begin{bmatrix} 0 \\ 0 \end{bmatrix}
\]


\end{homeworkProblem}

\end{document}
